% /home/alexis/git/robotic_introduction/CM/sources/intro_rob.tex
\documentclass{beamer}
\usetheme{Madrid}
\usecolortheme{dolphin}

\usepackage[utf8]{inputenc}
\usepackage[T1]{fontenc}
\usepackage{amsmath,amssymb}
\usepackage{graphicx}
\usepackage{siunitx}
\usepackage{tikz}
\usetikzlibrary{shapes,arrows,positioning}
\usepackage{booktabs}

\usepackage{ulem}

\title[Introduction à la robotique]{Introduction à la robotique}
\subtitle{De la robotique industriel à la cobotique}
\author{Alexis Boulay \\ \texttt{alexis.boulay@inria.fr}}
\date{2025/10/06}

\begin{document}

\begin{frame}
    \titlepage
\end{frame}


\begin{frame}{Détail de l'enseignement}
    \center
    
    \centering
    {
    \setlength{\tabcolsep}{12pt}%
    \renewcommand{\arraystretch}{1.6}%
    \begin{tabular}{@{} l c l @{}}
    \midrule
    CM & 2\,h & Introduction à la robotique \\
    TD & 4\,h & Modélisation robotique pour la commande \\
    TP & 12\,h & Programmation robotique d'un UR5 \\
    \bottomrule
    \end{tabular}
    }
    \vspace{0.7em}
    \pause
    \begin{block}{Objectif du cours}
        \sout{Faire de vous des roboticien·ne·s}

        Pouvoir discuter avec des roboticien·ne·s et intéragir avec des robots.        
    \end{block}

    \pause
    \begin{alertblock}{Evaluation}    \begin{itemize}
            \item Rapport de TP 
            \item Examen écrit (2H)
        \end{itemize}
    \end{alertblock}

\end{frame}

\begin{frame}{Outline}
    \tableofcontents
\end{frame}


\section{Qu'est-ce qu'un robot ?}

% \begin{frame}{Résumé}
%     \begin{enumerate}
%         \item Montrer plein de robot différents, demander de définir ce qu'est un robot
%         \item exemple perso quand je cherchais un emploi beaucoup de proposition de faire des chatbots.
%         \item Différencier automate et robot, donner les différentes définition ethymologique et préciser qu'il y a pas de définition forcément fixe
%         \item Donner les caractéristque d'un robot : degrés de liberté, type d'actionneur, link, EE, précision, répétabilité, charge de travail.
%     \end{enumerate}
% \end{frame}


\begin{frame}{Qu'est-ce qu'un robot ?}

    \begin{figure}
        \centering
        \scriptsize
        \begin{tabular}{@{}c@{\hspace{0.05cm}}c@{\hspace{0.05cm}}c@{\hspace{0.05cm}}c@{\hspace{0.05cm}}c@{}}
            \includegraphics[width=0.17\textwidth,height=0.18\textheight,keepaspectratio]{Pictures/Robots/spot.jpg} &
            \includegraphics[width=0.17\textwidth,height=0.18\textheight,keepaspectratio]{Pictures/Robots/kuka.jpg} &
            \includegraphics[width=0.17\textwidth,height=0.18\textheight,keepaspectratio]{Pictures/Robots/reachy.jpeg} &
            \includegraphics[width=0.17\textwidth,height=0.18\textheight,keepaspectratio]{Pictures/Robots/davinci.jpg} &
            \includegraphics[width=0.17\textwidth,height=0.18\textheight,keepaspectratio]{Pictures/Robots/scara.jpg} \\[0.05cm]
            \includegraphics[width=0.17\textwidth,height=0.18\textheight,keepaspectratio]{Pictures/Robots/NASA_Mars_Rover.jpg} &
            \includegraphics[width=0.17\textwidth,height=0.18\textheight,keepaspectratio]{Pictures/Robots/mavic.png} &
            \includegraphics[width=0.17\textwidth,height=0.18\textheight,keepaspectratio]{Pictures/Robots/oceanone.jpg} &
            \includegraphics[width=0.17\textwidth,height=0.18\textheight,keepaspectratio]{Pictures/Robots/amazon_robot.jpeg} &
            \includegraphics[width=0.12\textwidth,height=0.18\textheight,keepaspectratio]{Pictures/Robots/chatbot.jpeg} \\[0.05cm]
            \includegraphics[width=0.17\textwidth,height=0.18\textheight,keepaspectratio]{Pictures/Robots/neo_robot.jpg} &
            \includegraphics[width=0.17\textwidth,height=0.18\textheight,keepaspectratio]{Pictures/Robots/mixer02.jpg} &
            \includegraphics[width=0.17\textwidth,height=0.18\textheight,keepaspectratio]{Pictures/Robots/openduck.jpg} &
            \includegraphics[width=0.17\textwidth,height=0.18\textheight,keepaspectratio]{Pictures/Robots/exxact.jpg} &
            \includegraphics[width=0.17\textwidth,height=0.18\textheight,keepaspectratio]{Pictures/Robots/rob.png}
        \end{tabular}
    \end{figure}
\end{frame}


\begin{frame}
    \frametitle{Définition\only<2->{\textbf{S}}}
    
    \begin{block}{Définition de l'ancienne enseignante de ce cours}
    Un robot est un dispositif se comportant de manière automatique doté de capteurs et d'effecteurs lui donnant une capacité d'adaptation et de déplacement proche de l'autonomie. Un robot est un agent physique réalisant des tâches dans l'environnement dans lequel il évolue.
    \end{block}

    \pause

    \begin{block}{Définition d'un autre professeur de robotique}
        Système mécanique, poly-articulé, actionné, équipé de capteurs et dont le comportement est commandé automatiquement par un calculateur (re)programmable.
    \end{block}
\end{frame}

\begin{frame}{Caractéristiques d'un robot}
    \framesubtitle{L'architecture}

    \begin{itemize}
        \item Robot sériel (bras manipulateur, scara\ldots)
        \item Robot parallèle (robot delta, robot à câble\ldots)
        \item Robot hybride (humanoïde, quadrupède\ldots) 
        \item Robot mobile (terrestre ou aérien)
    \end{itemize}
    
    \vspace{0.5em}
    
    Type de moteurs : hydraulique, électrique, pneumatique
    
    \vspace{0.5em}
    
    \begin{figure}
        \centering
        \begin{tabular}{@{}c@{\hspace{0.15cm}}c@{\hspace{0.15cm}}c@{\hspace{0.15cm}}c@{}}
            \includegraphics[width=0.22\textwidth,height=0.28\textheight,keepaspectratio]{Pictures/Structures/seriel.png} &
            \includegraphics[width=0.22\textwidth,height=0.28\textheight,keepaspectratio]{Pictures/Structures/parallele.jpg} &
            \includegraphics[width=0.22\textwidth,height=0.28\textheight,keepaspectratio]{Pictures/Structures/hybride.jpg} &
            \includegraphics[width=0.22\textwidth,height=0.28\textheight,keepaspectratio]{Pictures/Structures/mobile.jpeg} \\
            \small Sérielle & \small Parallèle & \small Hybride & \small Mobile
        \end{tabular}
    \end{figure}
\end{frame}

\begin{frame}{Degrés de liberté et limites}

    \begin{block}{Degrés de liberté (DoF)}
        
        Le nombre de degrés de liberté (DoF) d'un robot sériel est le nombre d'actionneurs indépendants qu'il possède. Ce nombre définit la capacité de mouvement du robot dans son espace de travail.

    \end{block}

    \begin{figure}
        \centering
        \includegraphics[height=0.35\textheight]{Pictures/transfo_dof.png}
    \end{figure}

    \begin{table}
        \centering
        \scriptsize
        \begin{tabular}{@{}p{0.48\textwidth} p{0.48\textwidth}@{}}
            \textbf{Espace cartésien (monde)} & \textbf{Espace articulaire} \\ \hline\addlinespace[0.2em]
            Limite de l'espace de travail & Butées articulaires \\ \addlinespace[0.2em]
            Limites en vitesse et accélération cartésienne & Limites en vitesse et accélération des actionneurs \\
            Limites de charge & Limites de couple des actionneurs \\
        \end{tabular}
    \end{table}
\end{frame}

\begin{frame}{Caractéristiques d'un robot}
    \framesubtitle{Espace de travail et limites}
    \begin{figure}
        \centering
        \includegraphics[height=0.8\textheight]{Pictures/espace_travail.png}
    \end{figure}
\end{frame}

\begin{frame}{Caractéristiques d'un robot}
    \framesubtitle{Capacité de charge maximales et utiles}
    \begin{exampleblock}{Charges}
        \begin{description}
            \item[Charge maximale] : charge que le robot peut théoriquement soulever dans des conditions idéales.
            \item[Charge utile] : charge que le robot peut soulever en tenant compte de sa configuration, des contraintes dynamiques, de l'usure des composants et de la sécurité.
        \end{description}
    \end{exampleblock}

    \begin{figure}
        \centering
        \begin{tabular}{@{}c@{\hspace{0.5cm}}c@{\hspace{0.5cm}}c@{}}
            \includegraphics[width=0.28\textwidth,height=0.25\textheight,keepaspectratio]{Pictures/irb8700.png} &
            \includegraphics[width=0.28\textwidth,height=0.25\textheight,keepaspectratio]{Pictures/ur5.png} &
            \includegraphics[width=0.40\textwidth,height=0.30\textheight,keepaspectratio]{Pictures/ur5_payload.png} \\
            \small irb8700 (ABB) & \small UR 5e & \small UR 5e Charge utile \\
            Charge maximale 1000 kg &  & \\
            Charge utile 800 kg &  & \\
        \end{tabular}
    \end{figure}

\end{frame}

\begin{frame}{Caractéristiques d'un robot}
    \framesubtitle{Répétabilité et Précision}

    \begin{figure}
        \centering
        \includegraphics[width=0.4\textwidth]{Pictures/prec_rep.jpg}
    \end{figure}

    Les imprécisions d'un robot :
    \begin{columns}[t]
        \begin{column}{0.48\textwidth}
            \begin{itemize}
                \item Mauvaise calibration
                \item Mauvaise commande / modélisation
            \end{itemize}
        \end{column}
        \begin{column}{0.48\textwidth}
            \begin{itemize}
                \item Jeu dans les articulations
                \item Usure et déformation
            \end{itemize}
        \end{column}
    \end{columns}

\end{frame}

\begin{frame}{Caractéristiques d'un robot}
    \framesubtitle{Capteurs associés}

    \begin{table}
        \centering
        \scriptsize
        \begin{tabular}{|c|c|}
            \hline
            \textbf{Encodeur} & \textbf{Capteur de force} \\
            \includegraphics[width=0.4\textwidth,height=0.15\textheight,keepaspectratio]{Pictures/Sensors/encodeur.jpeg} & 
            \includegraphics[width=0.4\textwidth,height=0.15\textheight,keepaspectratio]{Pictures/Sensors/forcesensor.jpg} \\
            Position articulaire & Mesure des forces d'interaction \\
            \hline
            \textbf{LiDAR} & \textbf{Caméra 3D (Kinect)} \\
            \includegraphics[width=0.4\textwidth,height=0.15\textheight,keepaspectratio]{Pictures/Sensors/lidar.png} & 
            \includegraphics[width=0.4\textwidth,height=0.15\textheight,keepaspectratio]{Pictures/Sensors/kinect-v2.jpg} \\
            Cartographie et détection & Vision et perception 3D \\
            \hline
            \textbf{Barrière immatérielle} & \textbf{Cage de sécurité} \\
            \includegraphics[width=0.4\textwidth,height=0.15\textheight,keepaspectratio]{Pictures/Sensors/barriere-immaterielle.jpg} & 
            \includegraphics[width=0.4\textwidth,height=0.15\textheight,keepaspectratio]{Pictures/Sensors/cage.jpg} \\
            Détection de présence & Protection physique \\
            \hline
        \end{tabular}
    \end{table}

\end{frame}


\begin{frame}{Facteurs de choix d'un robot}
    \begin{enumerate}
        \item Type de robot (sériel, parallèle, mobile, hybride)
        \item Degrés de liberté nécessaires
        \item Espace de travail requis
        \item Capacité de charge utile
        \item Précision et répétabilité
        \item Vitesse et accélération
        \item Type d'actionneurs (électrique, hydraulique, pneumatique)
        \item Environnement de travail (intérieur, extérieur, conditions extrêmes)
    \end{enumerate}
    
\end{frame}

\section{La robotique à travers le temps}

% \begin{frame}{Plan de la section}
%     \begin{enumerate}
%         \item Le développement de la robotique industrielle (1950-1990)
%         \item La diversification des applications (1990-2010)
%         \item La robotique collaborative et l'IA (2010-aujourd'hui)
%         \item Normes et sécurité
%     \end{enumerate}
% \end{frame}

\begin{frame}{Le développement de la robotique industrielle}
    \begin{columns}
        \begin{column}{0.58\textwidth}
            \begin{itemize}
                \item<1-> \textbf{1961} : Unimate, premier robot industriel commercialisé (General Motors)
                \item<2-> \textbf{1973} : 1\textsuperscript{er} robot 6 axes Famulus (KUKA)
                \item<3-> \textbf{1985} : 1\textsuperscript{er} robot delta
                \item<4-> \textbf{2015} : Arrivée des robots collaboratifs
            \end{itemize}
        \end{column}
        
        \begin{column}{0.38\textwidth}
            \begin{figure}
                \centering
                \only<1>{\includegraphics[width=\textwidth,height=0.5\textheight,keepaspectratio]{Pictures/History/unimate.jpg}}
                \only<2>{\includegraphics[width=\textwidth,height=0.5\textheight,keepaspectratio]{Pictures/History/famulus.jpg}}
                \only<3>{\includegraphics[width=\textwidth,height=0.5\textheight,keepaspectratio]{Pictures/History/delta.jpg}}
                \only<4>{\includegraphics[width=\textwidth,height=0.5\textheight,keepaspectratio]{Pictures/ur5.png}}
            \end{figure}
        \end{column}
    \end{columns}
\end{frame}

\begin{frame}{La robotique collaborative}
    
    \begin{columns}
        \begin{column}{0.55\textwidth}
            \textbf{Les cobots (robots collaboratifs)}
            \begin{itemize}
                \item 2008 : Universal Robots UR5 (pionnier)
                \item Légers et sûrs pour travailler avec les humains
                \item Programmation intuitive
                \item Flexibles et réutilisables
            \end{itemize}
            
            \vspace{0.5em}
            
            \textbf{Autres applications collaboratives}
            \begin{itemize}
                \item Exosquelettes pour assistance physique
                \item Robots de rééducation
                \item Robots de service (hôtellerie, logistique)
            \end{itemize}
        \end{column}
        
        \begin{column}{0.42\textwidth}
            \begin{figure}
                \centering
                \includegraphics[width=\textwidth,height=0.6\textheight,keepaspectratio]{Pictures/ur5.png}
            \end{figure}
        \end{column}
    \end{columns}
\end{frame}

\begin{frame}{Robotique et Intelligence Artificielle}
    
    \begin{itemize}[<+->]
        \item \textbf{Apprentissage par démonstration}
        \begin{itemize}
            \item Le robot apprend en observant l'humain
            \item Réduction du temps de programmation
        \end{itemize}
        
        \item \textbf{Apprentissage par renforcement}
        \begin{itemize}
            \item Le robot apprend par essai-erreur
            \item Adaptation à des tâches complexes
            \item Peut être simulé
        \end{itemize}
        
        \item \textbf{VLA}
        \begin{itemize}
            \item Utilisation de LLM pour la comprehension de tâches
            \item Prise en charge de la Vision
            \item Apprentissage via démonstrations
        \end{itemize}
    \end{itemize}
\end{frame}

\begin{frame}{Les normes et la sécurité}
    \framesubtitle{Encadrement de la robotique}
    
    \begin{exampleblock}{Principales normes}
        \begin{itemize}
            \item \textbf{ISO 10218-1/2} : Robots industriels - Exigences de sécurité
            \item \textbf{ISO/TS 15066} : Robots collaboratifs - Spécifications de sécurité
            \item \textbf{ISO 13482} : Robots de service - Exigences de sécurité
        \end{itemize}
    \end{exampleblock}
    
    \vspace{0.5em}
    
    \textbf{Principes clés :}
    \begin{itemize}
        \item Analyse des risques
        \item Limitation de force et de puissance pour les cobots
        \item Systèmes de sécurité (arrêts d'urgence, barrières)
        \item Formation des opérateurs
        \item Maintenance préventive
    \end{itemize}
\end{frame}

\section{La robotique de nos jours}

% \begin{frame}{Plan de la section}
%     \begin{enumerate}
%         \item Le marché mondial de la robotique
%         \item Utilisation par pays et régions
%         \item Domaines d'application
%         \item Cas d'usage concrets et vidéos
%         \item Défis et opportunités
%     \end{enumerate}
% \end{frame}

\begin{frame}{Quelques chiffres}
    \begin{figure}
        \centering
        \includegraphics[width=\textwidth,height=0.9\textheight,keepaspectratio]{Pictures/Nowadays/chiffre_robot1.png}
    \end{figure}
\end{frame}

\begin{frame}{Quelques chiffres}
    \begin{figure}
        \centering
        \includegraphics[width=\textwidth,height=0.9\textheight,keepaspectratio]{Pictures/Nowadays/chiffre_robot2.png}
    \end{figure}
\end{frame}

\begin{frame}{Quelques chiffres}
    \begin{figure}
        \centering
        \includegraphics[width=\textwidth,height=0.9\textheight,keepaspectratio]{Pictures/Nowadays/chiffre_robot3.png}
    \end{figure}
\end{frame}

\begin{frame}{Quelques chiffres}
    \begin{figure}
        \centering
        \includegraphics[width=\textwidth,height=0.9\textheight,keepaspectratio]{Pictures/Nowadays/chiffre_robot4.png}
    \end{figure}
\end{frame}

\begin{frame}{Quelques chiffres}
    \begin{figure}
        \centering
        \includegraphics[width=\textwidth,height=0.9\textheight,keepaspectratio]{Pictures/Nowadays/chiffre_robot5.png}
    \end{figure}
\end{frame}

\begin{frame}{Quelques chiffres}
    \begin{figure}
        \centering
        \includegraphics[width=\textwidth,height=0.9\textheight,keepaspectratio]{Pictures/Nowadays/chiffre_robot6.png}
    \end{figure}
\end{frame}

\begin{frame}{Quelques chiffres}
    \begin{figure}
        \centering
        \includegraphics[width=\textwidth,height=0.9\textheight,keepaspectratio]{Pictures/Nowadays/chiffre_robot7.png}
    \end{figure}
\end{frame}


\begin{frame}{Au-delà de l'industrie traditionnelle}
    
    \begin{itemize}[<+->]
        \item \textbf{Logistique et e-commerce}
        \begin{itemize}
            \item Préparation de commandes
            \item Tri et emballage
        \end{itemize}
        
        \item \textbf{Santé et médical}
        \begin{itemize}
            \item Chirurgie assistée (da Vinci)
            \item Réhabilitation et rééducation
        \end{itemize}
        
        \item \textbf{Agriculture}
        \begin{itemize}
            \item Robots de récolte
            \item Surveillance des cultures (drones)
        \end{itemize}
        
        \item \textbf{Construction}
        \begin{itemize}
            \item Impression 3D de bâtiments
            \item Robots de maçonnerie
        \end{itemize}
        
        \item \textbf{Services}
        \begin{itemize}
            \item Nettoyage (hôtels, aéroports)
            \item Livraison urbaine
            \item Accueil et information
        \end{itemize}
        
        \item \textbf{Exploration}
        \begin{itemize}
            \item Intervention en milieu hostile
        \end{itemize}
    \end{itemize}
\end{frame}

\begin{frame}{Cas d'usage : Logistique Amazon}
    
    \begin{columns}
        \begin{column}{0.55\textwidth}
            \textbf{Robots Kiva/Amazon Robotics}
            \begin{itemize}
                \item +750 000 robots déployés
                \item Réduction du temps de préparation de 50\%
                \item Optimisation de l'espace d'entrepôt
                \item Collaboration humain-robot
            \end{itemize}
            
            \vspace{0.5em}
            
            \textbf{Technologies utilisées}
            \begin{itemize}
                \item Navigation autonome
                \item Vision par ordinateur
                \item Coordination multi-robots
                \item IA pour optimisation de flux
            \end{itemize}
        \end{column}
        
        \begin{column}{0.42\textwidth}
            \begin{figure}
                \centering
                \includegraphics[width=\textwidth,height=0.6\textheight,keepaspectratio]{Pictures/Robots/amazon_robot.jpeg}
            \end{figure}
        \end{column}
    \end{columns}
\end{frame}

\begin{frame}{Cas d'usage : Boston Dynamics Spot}
    
    \begin{columns}
        \begin{column}{0.55\textwidth}
            \textbf{Applications}
            \begin{itemize}
                \item Inspection industrielle (pétrole, gaz)
                \item Surveillance de sites
                \item Cartographie 3D
                \item Intervention en zone dangereuse
            \end{itemize}
            
            \vspace{0.5em}
            
            \textbf{Capacités}
            \begin{itemize}
                \item Mobilité sur terrain accidenté
                \item Navigation autonome
                \item Équipement modulaire (caméras, capteurs)
                \item Opération à distance
            \end{itemize}
        \end{column}
        
        \begin{column}{0.42\textwidth}
            \begin{figure}
                \centering
                \includegraphics[width=\textwidth,height=0.6\textheight,keepaspectratio]{Pictures/Robots/spot.jpg}
                \caption*{\small Boston Dynamics Spot}
            \end{figure}
        \end{column}
    \end{columns}
\end{frame}

\begin{frame}{Cas d'usage : Chirurgie robotique}
    
    \begin{columns}
        \begin{column}{0.52\textwidth}
            \textbf{Système da Vinci}
            \begin{itemize}
                \item +7 000 systèmes installés
                \item +10 millions d'interventions
                \item Chirurgie mini-invasive
            \end{itemize}
            
            \vspace{0.5em}
            
            \textbf{Avantages}
            \begin{itemize}
                \item Précision submillimétrique
                \item Vision 3D HD
                \item Réduction des tremblements
                \item Récupération plus rapide
                \item Moins de complications
            \end{itemize}
        \end{column}
        
        \begin{column}{0.45\textwidth}
            \begin{figure}
                \centering
                \includegraphics[width=\textwidth,height=0.6\textheight,keepaspectratio]{Pictures/Robots/davinci.jpg}
                \caption*{\small Robot da Vinci}
            \end{figure}
        \end{column}
    \end{columns}
\end{frame}

\begin{frame}{Opportunités et perspectives}
    \framesubtitle{La robotique collaborative}
    
    \begin{figure}
        \centering
        \includegraphics[width=\textwidth,height=0.9\textheight,keepaspectratio]{Pictures/IFR_collaboration.jpg}
    \end{figure}
\end{frame}

\begin{frame}{Robotique et VLA (Vision-Language-Action)}
    \framesubtitle{La nouvelle génération de robots intelligents}

    \begin{columns}
        \begin{column}{0.58\textwidth}
            \scriptsize
            \textbf{Concept VLA}
            \begin{itemize}
                \item Vision + Language + Action
                \item Instructions en langage naturel
                \item Apprentissage par imitation
            \end{itemize}
            
            \vspace{0.3em}
            
            \textbf{Exemples de robots}
            \begin{itemize}
                \item \textbf{1X Neo}: Robot domestique
                \item \textbf{Google RT-2 + Gemini}: Raisonnement avancé
                \item \textbf{Figure AI}: Humanoïde avec OpenAI
                \item \textbf{Tesla Optimus}: Production de masse
            \end{itemize}
        \end{column}
        
        \begin{column}{0.38\textwidth}
            \begin{figure}
                \centering
                \includegraphics[width=\textwidth,height=0.5\textheight,keepaspectratio]{Pictures/Robots/neo_robot.jpg}
            \end{figure}
        \end{column}
    \end{columns}
\end{frame}

\begin{frame}{Quels usages pour la robotique dans un monde fini et incertain ?}

    \begin{alertblock}{Est-ce qu'on veut des robots partout ?}
        \begin{itemize}
            \item Impact environnemental de la production et de l'utilisation des robots
            \item Consommation énergétique
            \item Gestion des déchets électroniques
            \item Éthique et acceptabilité sociale
            \item La place de l'humain dans un monde automatisé ?
        \end{itemize}
        
    \end{alertblock}

\end{frame}


\section{Les problématiques de la robotique}

% \begin{frame}{Résumé}
%     \begin{itemize}
%         \item Différencier les acteurs de la robotique: de la conception à l'intégration en passant par la commande (Vincent à une slide sur les problèmes de la robotique)
%         \item zoom sur ce qui va nous intéresser la modélisation géométrique et cinématique, programmation et les différentes méthodes d'utilisation de la robotique.
%     \end{itemize}
% \end{frame}

\begin{frame}{Problématiques de la robotique}
    \begin{itemize}[<+->]
        \item \alert<1>{Conception (mécanique et mécatronique) -- typologie, morphologie, DoFs, dimensionnement}
        \item \alert<2>{Identification et calibration -- hors ligne et en ligne}
        \item \alert<3>{Perception et estimation de l'état -- usages de capteurs internes et externes}
        \item \alert<4>{Planification de trajectoire -- hors ligne et en ligne}
        \item \alert<5>{Commande -- compromis robustesse/performance/sécurité}
        \item \alert<6>{Interaction humain-robot -- tâche partagée, apprentissage par démonstration\ldots}
        \item \alert<7>{Intégration dans un environnement industriel}
        \item \alert<8>{Process métier, outillage}
        \item \alert<9>{\ldots}
    \end{itemize}
\end{frame}

\begin{frame}{Zoom sur la commande}
    \framesubtitle{Boucle de commande d'un robot}

    \begin{figure}
        \centering
        \includegraphics[width=\textwidth]{Pictures/commande.png}
    \end{figure}

    \begin{itemize}
        \item Dépendance au robot : pas accès au même type de commande et de mesures.
        \item Dépendance à la tâche : pas les mêmes besoins en terme de contrôle.
    \end{itemize}

\end{frame}

\begin{frame}{Zoom sur la commande}
    \framesubtitle{Passage du monde cartésien au monde articulaire}

    \begin{exampleblock}{Modèle géométrique directe}
        Permet de passer d'une position articulaire ($\mathbb{R}^n$, avec $n$ le nombre d'articulations) à une position cartésienne ($\mathbb{R}^6$).
    \end{exampleblock}

    \begin{minipage}
        {0.45\textwidth}
        \centering
        \includegraphics[width=0.7\textwidth]{Pictures/3R.png}
    \end{minipage}
    \hfill
    \begin{minipage}
        {0.45\textwidth}
        Espace articulaire : \textcolor{blue}{$(q_1, q_2, q_3)$} $\in \mathbb{R}^3$

        Espace cartésien : $\begin{Bmatrix}\begin{pmatrix}x_{EE} \\ y_{EE} \\ \theta_{EE}\end{pmatrix}_{w} R_{EE\in w}\end{Bmatrix}$
    \end{minipage}
\end{frame}

\begin{frame}{Zoom sur la commande}
    \framesubtitle{Passage du monde articulaire au monde cartésien}

    \begin{exampleblock}{Modèle géométrique indirecte}
        Permet de passer d'une position cartésienne ($\mathbb{R}^6$) à une position articulaire ($\mathbb{R}^n$, avec $n$ le nombre d'articulations).
    \end{exampleblock}

    \begin{minipage}
        {0.45\textwidth}
        \centering
        \includegraphics[width=0.7\textwidth]{Pictures/3R.png}
    \end{minipage}
    \hfill
    \begin{minipage}
        {0.45\textwidth}
        
        Il est nécessaire de bien définir nos repères pour définir la position cartésienne.
        Il existe plusieurs conventions (nous verrons Denavit-Hartenberg en TD).

    \end{minipage}

Plus utile pour la commande, mais plus compliqué à calculer.
\end{frame}


\begin{frame}{Zoom sur la commande}

    \begin{exampleblock}{Modèle cinématique}
        Fait le lien entre vitesse articulaire et vitesse cartésienne.
    \end{exampleblock}

    \vspace{0.5cm}

    \begin{minipage}
        {0.45\textwidth}
        \centering
        \includegraphics[width=0.4\textwidth]{Pictures/3R_cin.png}
    \end{minipage}
    \hfill
    \begin{minipage}
        {0.45\textwidth}

        $\dot{\mathbf{X}} = \mathbf{J}(\mathbf{q}) \dot{\mathbf{q}}$

        La matrice $\mathbf{J}(\mathbf{q})$ est appelé la jacobienne du robot.

        $\mathbf{J}$ inversible $\Rightarrow$ on peut faire du contrôle en vitesse.

        $\mathbf{J} $ non inversible $\Rightarrow$ on est en présence de singularités.
    \end{minipage}

    \pause
    \vspace{0.5cm}

    \begin{exampleblock}{Modèle dynamique}
        Passer de forces/torques cartésiennes à des couples articulaires.
    \end{exampleblock}

\end{frame}
    

\begin{frame}{Zoom sur la planification}
    \framesubtitle{Génération de trajectoire}

    \begin{columns}
        \begin{column}{0.48\textwidth}
            \small
            \textbf{Cas 1 : Planification automatique}
            Peut être fait hors ligne ou en ligne.
            \begin{enumerate}
                \item Cible cartésienne
                \item Algorithme (RRT, A*, etc.)
                \item Conversion articulaire
            \end{enumerate}
            
            \vspace{0.5em}
            
            \textbf{Cas 2 : Programmation manuelle}
            Hors ligne uniquement, plus courant en industrie
            \begin{enumerate}
                \item Enregistrement de configurations/trajectoires articulaires
                \item Rejeu de la trajectoire
            \end{enumerate}
        \end{column}

        \begin{column}{0.48\textwidth}
            \begin{figure}
                \centering
                \includegraphics[width=\textwidth,height=0.6\textheight,keepaspectratio]{Pictures/RRT.png}
            \end{figure}
        \end{column}
    \end{columns}
\end{frame}

\end{document}