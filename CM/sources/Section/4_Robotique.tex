\section{Les problématiques de la robotique}

% \begin{frame}{Résumé}
%     \begin{itemize}
%         \item Différencier les acteurs de la robotique: de la conception à l'intégration en passant par la commande (Vincent à une slide sur les problèmes de la robotique)
%         \item zoom sur ce qui va nous intéresser la modélisation géométrique et cinématique, programmation et les différentes méthodes d'utilisation de la robotique.
%     \end{itemize}
% \end{frame}

\begin{frame}{Problématiques de la robotique}
    \begin{itemize}[<+->]
        \item \alert<1>{Conception (mécanique et mécatronique) -- typologie, morphologie, DoFs, dimensionnement}
        \item \alert<2>{Identification et calibration -- hors ligne et en ligne}
        \item \alert<3>{Perception et estimation de l'état -- usages de capteurs internes et externes}
        \item \alert<4>{Planification de trajectoire -- hors ligne et en ligne}
        \item \alert<5>{Commande -- compromis robustesse/performance/sécurité}
        \item \alert<6>{Interaction humain-robot -- tâche partagée, apprentissage par démonstration\ldots}
        \item \alert<7>{Intégration dans un environnement industriel}
        \item \alert<8>{Process métier, outillage}
        \item \alert<9>{\ldots}
    \end{itemize}
\end{frame}

\begin{frame}{Zoom sur la commande}
    \framesubtitle{Boucle de commande d'un robot}

    \begin{figure}
        \centering
        \includegraphics[width=\textwidth]{Pictures/commande.png}
    \end{figure}

    \begin{itemize}
        \item Dépendance au robot : pas accès au même type de commande et de mesures.
        \item Dépendance à la tâche : pas les mêmes besoins en terme de contrôle.
    \end{itemize}

\end{frame}

\begin{frame}{Zoom sur la commande}
    \framesubtitle{Passage du monde cartésien au monde articulaire}

    \begin{exampleblock}{Modèle géométrique directe}
        Permet de passer d'une position articulaire ($\mathbb{R}^n$, avec $n$ le nombre d'articulations) à une position cartésienne ($\mathbb{R}^6$).
    \end{exampleblock}

    \begin{minipage}
        {0.45\textwidth}
        \centering
        \includegraphics[width=0.7\textwidth]{Pictures/3R.png}
    \end{minipage}
    \hfill
    \begin{minipage}
        {0.45\textwidth}
        Espace articulaire : \textcolor{blue}{$(q_1, q_2, q_3)$} $\in \mathbb{R}^3$

        Espace cartésien : $\begin{Bmatrix}\begin{pmatrix}x_{EE} \\ y_{EE} \\ \theta_{EE}\end{pmatrix}_{w} R_{EE\in w}\end{Bmatrix}$
    \end{minipage}
\end{frame}

\begin{frame}{Zoom sur la commande}
    \framesubtitle{Passage du monde articulaire au monde cartésien}

    \begin{exampleblock}{Modèle géométrique indirecte}
        Permet de passer d'une position cartésienne ($\mathbb{R}^6$) à une position articulaire ($\mathbb{R}^n$, avec $n$ le nombre d'articulations).
    \end{exampleblock}

    \begin{minipage}
        {0.45\textwidth}
        \centering
        \includegraphics[width=0.7\textwidth]{Pictures/3R.png}
    \end{minipage}
    \hfill
    \begin{minipage}
        {0.45\textwidth}
        
        Il est nécessaire de bien définir nos repères pour définir la position cartésienne.
        Il existe plusieurs conventions (nous verrons Denavit-Hartenberg en TD).

    \end{minipage}

Plus utile pour la commande, mais plus compliqué à calculer.
\end{frame}


\begin{frame}{Zoom sur la commande}

    \begin{exampleblock}{Modèle cinématique}
        Fait le lien entre vitesse articulaire et vitesse cartésienne.
    \end{exampleblock}

    \vspace{0.5cm}

    \begin{minipage}
        {0.45\textwidth}
        \centering
        \includegraphics[width=0.4\textwidth]{Pictures/3R_cin.png}
    \end{minipage}
    \hfill
    \begin{minipage}
        {0.45\textwidth}

        $\dot{\mathbf{X}} = \mathbf{J}(\mathbf{q}) \dot{\mathbf{q}}$

        La matrice $\mathbf{J}(\mathbf{q})$ est appelé la jacobienne du robot.

        $\mathbf{J}$ inversible $\Rightarrow$ on peut faire du contrôle en vitesse.

        $\mathbf{J} $ non inversible $\Rightarrow$ on est en présence de singularités.
    \end{minipage}

    \pause
    \vspace{0.5cm}

    \begin{exampleblock}{Modèle dynamique}
        Passer de forces/torques cartésiennes à des couples articulaires.
    \end{exampleblock}

\end{frame}
    

\begin{frame}{Zoom sur la planification}
    \framesubtitle{Génération de trajectoire}

    \begin{columns}
        \begin{column}{0.48\textwidth}
            \small
            \textbf{Cas 1 : Planification automatique}
            Peut être fait hors ligne ou en ligne.
            \begin{enumerate}
                \item Cible cartésienne
                \item Algorithme (RRT, A*, etc.)
                \item Conversion articulaire
            \end{enumerate}
            
            \vspace{0.5em}
            
            \textbf{Cas 2 : Programmation manuelle}
            Hors ligne uniquement, plus courant en industrie
            \begin{enumerate}
                \item Enregistrement de configurations/trajectoires articulaires
                \item Rejeu de la trajectoire
            \end{enumerate}
        \end{column}

        \begin{column}{0.48\textwidth}
            \begin{figure}
                \centering
                \includegraphics[width=\textwidth,height=0.6\textheight,keepaspectratio]{Pictures/RRT.png}
            \end{figure}
        \end{column}
    \end{columns}
\end{frame}