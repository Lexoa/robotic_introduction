\section{La robotique à travers le temps}

% \begin{frame}{Plan de la section}
%     \begin{enumerate}
%         \item Le développement de la robotique industrielle (1950-1990)
%         \item La diversification des applications (1990-2010)
%         \item La robotique collaborative et l'IA (2010-aujourd'hui)
%         \item Normes et sécurité
%     \end{enumerate}
% \end{frame}

\begin{frame}{Le développement de la robotique industrielle}
    \begin{columns}
        \begin{column}{0.58\textwidth}
            \begin{itemize}
                \item<1-> \textbf{1961} : Unimate, premier robot industriel commercialisé (General Motors)
                \item<2-> \textbf{1973} : 1\textsuperscript{er} robot 6 axes Famulus (KUKA)
                \item<3-> \textbf{1985} : 1\textsuperscript{er} robot delta
                \item<4-> \textbf{2015} : Arrivée des robots collaboratifs
            \end{itemize}
        \end{column}
        
        \begin{column}{0.38\textwidth}
            \begin{figure}
                \centering
                \only<1>{\includegraphics[width=\textwidth,height=0.5\textheight,keepaspectratio]{Pictures/History/unimate.jpg}}
                \only<2>{\includegraphics[width=\textwidth,height=0.5\textheight,keepaspectratio]{Pictures/History/famulus.jpg}}
                \only<3>{\includegraphics[width=\textwidth,height=0.5\textheight,keepaspectratio]{Pictures/History/delta.jpg}}
                \only<4>{\includegraphics[width=\textwidth,height=0.5\textheight,keepaspectratio]{Pictures/ur5.png}}
            \end{figure}
        \end{column}
    \end{columns}
\end{frame}

\begin{frame}{La robotique collaborative}
    
    \begin{columns}
        \begin{column}{0.55\textwidth}
            \textbf{Les cobots (robots collaboratifs)}
            \begin{itemize}
                \item 2008 : Universal Robots UR5 (pionnier)
                \item Légers et sûrs pour travailler avec les humains
                \item Programmation intuitive
                \item Flexibles et réutilisables
            \end{itemize}
            
            \vspace{0.5em}
            
            \textbf{Autres applications collaboratives}
            \begin{itemize}
                \item Exosquelettes pour assistance physique
                \item Robots de rééducation
                \item Robots de service (hôtellerie, logistique)
            \end{itemize}
        \end{column}
        
        \begin{column}{0.42\textwidth}
            \begin{figure}
                \centering
                \includegraphics[width=\textwidth,height=0.6\textheight,keepaspectratio]{Pictures/ur5.png}
            \end{figure}
        \end{column}
    \end{columns}
\end{frame}

\begin{frame}{Robotique et Intelligence Artificielle}
    
    \begin{itemize}[<+->]
        \item \textbf{Apprentissage par démonstration}
        \begin{itemize}
            \item Le robot apprend en observant l'humain
            \item Réduction du temps de programmation
        \end{itemize}
        
        \item \textbf{Apprentissage par renforcement}
        \begin{itemize}
            \item Le robot apprend par essai-erreur
            \item Adaptation à des tâches complexes
            \item Peut être simulé
        \end{itemize}
        
        \item \textbf{VLA}
        \begin{itemize}
            \item Utilisation de LLM pour la comprehension de tâches
            \item Prise en charge de la Vision
            \item Apprentissage via démonstrations
        \end{itemize}
    \end{itemize}
\end{frame}

\begin{frame}{Les normes et la sécurité}
    \framesubtitle{Encadrement de la robotique}
    
    \begin{exampleblock}{Principales normes}
        \begin{itemize}
            \item \textbf{ISO 10218-1/2} : Robots industriels - Exigences de sécurité
            \item \textbf{ISO/TS 15066} : Robots collaboratifs - Spécifications de sécurité
            \item \textbf{ISO 13482} : Robots de service - Exigences de sécurité
        \end{itemize}
    \end{exampleblock}
    
    \vspace{0.5em}
    
    \textbf{Principes clés :}
    \begin{itemize}
        \item Analyse des risques
        \item Limitation de force et de puissance pour les cobots
        \item Systèmes de sécurité (arrêts d'urgence, barrières)
        \item Formation des opérateurs
        \item Maintenance préventive
    \end{itemize}
\end{frame}